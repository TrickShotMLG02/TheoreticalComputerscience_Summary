\sectionwithtitle{Regular Expressions}
    A regular expression always describes a regular language. If we can build a regular expression E, then L(E) \elem \REG. \newline
    
    \subsectionwithtitle{Languages of regular expressions}
    If $E = \emptyset$ then $L(E) = \emptyset$\newline
    If $E = \emptyword$ then $L(E) = \{\emptyword\}$\newline
    If $E = \symbol$ then $L(E) = \{ \symbol \}$\newline
    If $E = (E_1 + E_2)$ then $L(E) = L(E_1) \cup L(E_2)$\newline
    If $E = (E_1 E_2)$ then $L(E) = L(E_1) L(E_2)$\newline
    If $E = (E_1)^*$ then $L(E) = L(E_1)^*$\newline

    \subsectionwithtitle{Identities}
    \begin{enumerate}
        \item $E + F = F + E$
        \item $(E + F) + G = E + (F + G)$
        \item $(EF)G = E(FG)$
        \item $\emptyset + E = E + \emptyset = E$
        \item $\emptyword E = E \emptyword = E$
        \item $\emptyset E = E \emptyset = \emptyset$
        \item $E + E = E$
        \item $(E + F)G = (EG) + (FG)$
        \item $E(F + G) = (EF) + (EG)$
        \item $(E^*)^* = E^*$
        \item $\emptyset ^* = \emptyword$
        \item $\emptyword^* = \emptyword$
    \end{enumerate}

    \subsectionwithtitle{Regular expressions and finite automata}
    If $E$ is a regular expression, then $L(E) \elem \REG$.\newline
    For every deterministic finite automaton $M = (Q, \sumS, \delta, q_0, Q_{acc})$ there is a regular
    expression $E$ with $L(M) = L(E)$.\newline