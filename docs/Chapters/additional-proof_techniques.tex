\sectionwithtitle{Common Proof Techniques}
    \subsectionwithtitle{Pumping Lemma}
    The pumping lemma can only be used to show that a language is not \REG. We take a word \word \elem $L$ and split it into multiple sub words $u,v,w$
    where $\abs{v}\geq n$ with $n \ge 0$. Now there are words $x,y,z$ with $v=xyz$ and $\abs{y} \ge 0$ such that $uxy^izw$ \elem $L$ for all
    $i$ \elem \NN. Afterwards we try to find an $i$ such that the pumped word is no longer \elem $L$ and thus we proved that the language is not regular.

    \subsubsectionwithtitle{Example}
    \textbf{Exercise}:\newline
    Show that the following language is not regular.\newline
    Let $A=\{ 1^{(3n)} \mid n \elem \NN \}$\newline\newline
    \textbf{Solution}:\newline
    Let $n \geq 0$ be given.\newline
    We choose $u=1^n$, $v=1^n$, $w=1^n$ such that $uvw=1^{3n}$ and $\abs{v}=n$.\newline
    Let $x,y,z$ be given as $x=1^r$, $y=1^s$, $z=1^t$ with $xyz=v$ and $s \ge 0$ since $y \neq \emptyword$ and $r+s+t=n$.\newline
    $uxy^izw=1^n 1^r 1^{s \cdot i} 1^t 1^n=1^n 1^{r+s+t} 1^{s \cdot (i-1)} 1^n$\newline
    We choose $i=0$, therefore $1^n 1^{r+s+t} 1^{s \cdot (i-1)} 1^n = 1^n 1^{n-s} 1^n$ \notelem $A$ since it is not of the form $1^{3m}$ anymore for any $m$ \elem \NN.\newline
    Therefore we cannot pump language $A$ and thus it is not regular.\newline

    \subsectionwithtitle{Myhill Nerode}
    \subsubsectionwithtitle{Example}
    \todo{Add example for Myhill Nerode}

    \newpage

    \subsectionwithtitle{Reductions}
    \subsubsectionwithtitle{Example}
    \textbf{Exercise}:\newline
    Consider the following language $L = \{ i \elem \NN \mid \text{göd}^{-1}(i) \text{ outputs 42 on input 1337} \}$\newline
    Prove $L \notelem \coRE$\newline\newline
    \textbf{General Ideas}:\newline
    $\complement{L}$ is the set of all programs that either diverge or output something else than 42 on input 1337.\newline
    $L$ is the set of all programs that output 42 on input 1337 which is equal to running a sub program that halts on every input and outputting 42 afterwards.
    Thus we could try to reduce the special halting problem $H_0$ to $L$.\newline\newline
    \textbf{Solution}:\newline
    $\forall L: L \elem \REC \iff L \elem \RE \wedge \complement{L} \elem \RE$\newline
    $L \notelem \coRE \iff \complement{L} \elem \RE$\newline\newline
    $H_0 \notelem \REC$ but $H_0 \elem \RE$, therefore $\complement{H_0} \notelem \RE$ thus reducing $\complement{H_0} \leq \complement{L}$ suffices to show that
    $\complement{L} \notelem \RE$ but this is the same as $H_0 \leq L$ by equivalence.\newline\newline
    The reduction is given by $f(i) := \text{göd}(P)$\newline
    We create the WHILE program $P$, that has input $m$. However we ignore the input $m$ and simulate $i$ on the input $i$, then return 42.
    Thus $f$ is obviously WHILE computable.\newline\newline
    We consider the following two cases:
    \begin{itemize}
        \item Let $i \elem H_0$, then $i$ halts on input $i$ by definition. Program $P$ halts on arbitrary inputs m and outputs 42.
        Therefore $P$ also outputs 42 on our input 1337.\newline
        Thus $\text{göd}(P) \elem L$ holds.
        \item Let $i \notelem H_0$, then $i$ does not halt on input $i$ by definition. Therefore Program $P$ diverges on all inputs, which is also the case for input 1337.\newline
        Therefore $P$ diverges on our input 1337.\newline
        Thus $\text{göd}(P) \notelem L$ holds.
    \end{itemize}
    Therefore we have successfully shown that $H_0 \leq L$ is a valid reduction.