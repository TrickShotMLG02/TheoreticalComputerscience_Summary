\documentclass[10pt,a4paper,english]{article}

\def\headersdir{../headers/}
\input{\headersdir udstasksheet.tex}

\lecture{\texttt{udstasksheet} -- Documentation}
\author{Robert Pietsch (\href{mailto:robert.pietsch@uni-saarland.de}{robert.pietsch@uni-saarland.de})}

\begin{document}
    \maketitle

    % A named exercise
    \ex[Getting Started]
    The easiest way to use this template is to copy the fitting example code from the \verb|examples/| folder.
    In there, you will find three different variants for different ways of submitting tasks:

    \begin{itemize}
        \item \verb|all-exercises-in-one-file| The whole source code for each task sheet is packed into a single \LaTeX~source file.
        Recommended if you submit your sheets alone and all tasks should be submitted in one PDF file.
        \item \verb|exercises-in-split-files| The source code is split into a seperate file for each task but all tasks are compiled into a single PDF file.
        Recommended if you solve sheets in a group (especially when using Git) and all tasks should be submitted in one PDF file. 
        \item \verb|pdf-per-exercise| Both the source code and the PDF output is split by task into seperate files.
        Recommended if task sheets should be submitted with a PDF per task instead per sheet.
    \end{itemize}

    Once you picked the right example code, you have to change a few lines to it to be ready for use.

    First of all, you should set the language for your submission.
    To do so, enter the correct argument (\verb|ngerman| or \verb|english|) for the \verb|\documentclass| command.
    \begin{minted}{latex}
        \documentclass[10pt,a4paper,ngerman]{article} % For submissions in German
        \documentclass[10pt,a4paper,english]{article} % For submissions in English
    \end{minted}

    Having done this, you have to set the correct path to the headers directory. 
    You can do so by changing the following line.
    Remember that \verb|../| means \enquote{go up to the parent folder}.
    \begin{minted}{latex}
        \def\headersdir{../../headers/}
    \end{minted}

    You should also enter some information for the title and page headers.
    These include the name of the lecture (\verb|\lecture|), 
    the task sheet number (\verb|\setsheetno|, remove to hide the sheet number), 
    the due date (\verb|\date|, remove to hide due date),
    and the authors (\verb|\author|). 
    Make sure to put these in the preamble (the code before \verb|\begin{document}|).
    \begin{minted}{latex}
        \lecture{Foundation of \LaTeX}
        \setsheetno{1}
        \date{March 13}
        \author{Dr. Herrmann Einstein (4201337)}
    \end{minted}

    With all of this set, you are ready to write down your solutions.
    Have fun!

    \pagebreak
    \ex[Special Task Sheet Macros]
    To mark the beginning of a new task, subtask, or sub-subtask use the following macros.
    You can specify a title for the given ((sub-)sub-)task by putting it in square brackets after the respective macro.
    \begin{minted}{latex}
        \ex
        \subex
        \subsubex

        \ex[An awesome title]
        \subex[An even more awesome title]
        \subsubex[The most awesome title]
    \end{minted}

    To change the style of numbering for the sheet or for tasks, you can use the following macros.
    These macros can also be used for ((sub-)sub-)tasks by replacing the \verb|\sheet| part inside the macros by \verb|\ex|, \verb|\subex|, or \verb|\subsubex|, respectivly.
    \begin{minted}{latex}
        \sheetstylearabic   % Arabic numerals (1, 2, 3, ...)
        \sheetstyleroman    % Lowercase roman numerals (i, ii, iii, ...)
        \sheetstyleRoman    % Uppercase roman numerals (I, II, III, ...)
        \sheetstylealph     % Lowercase latin alphabet (a, b, c, d)
        \sheetstyleAlph     % Uppercase latin alphabet (A, B, C, D)
    \end{minted}

    To change the number of a sheet or task, you can use the following macros.
    Independent of the numbering style, the new number has to be provided in arabic numerals.
    \begin{minted}{latex}
        \setsheetno{1}
        \setexno{3}
        \setsubexno{3}
        \setsubsubexno{7}
    \end{minted}

    \pagebreak
    \ex[Typesetting Math]
    \newcommand{\coderow}[2]{\mintinline{latex}{#1} & $#1$ & #2 \\}
    To make typesetting math more convenient, this template provides a plethora of macros that can be used in math mode.
    
    \begin{table}[H]
        \centering
        \begin{tabular}{@{}lcl@{}}
        \toprule
        Source Code & Result & Description \\ \midrule
        \coderow{\dif x}{d for integrals and differentials}
        \coderow{\im x}{Imaginary part}
        \coderow{\re x}{Real part}
        \coderow{45 \dgr}{Degree sign}
        \coderow{x \inv}{Inverse}
        \coderow{x \tx{3}}{Power of ten (scientific notation)}
        \coderow{\abs{x}}{Absolute value}
        \coderow{\brk{x}}{Round brackets}
        \coderow{\sbr{x}}{Square brackets}
        \coderow{\ceil{x}}{Euclidian round-up brackets}
        \coderow{\floor{x}}{Euclidian round-down brackets}
        \coderow{\set{x}}{Set}
        \coderow{\set{x \where y}}{Set with condition}
        \coderow{x \bdiv y}{Integer division}
        \coderow{x \bmod y}{Integer division remainder}
        \coderow{\vcol}{Vertically centered colon}
        \coderow{\defeq}{Definition operator}
        \coderow{\eqdef}{Mirrored definition operator}
        \midrule
        \coderow{\NN}{Set of Natural Numbers}
        \coderow{\ZZ}{Set of Integers}
        \coderow{\QQ}{Set of Rational Numbers}
        \coderow{\RR}{Set of Real Numbers}
        \coderow{\CC}{Set of Complex Numbers}
        \coderow{\PP}{Set of Prime Numbers}
        \coderow{\HH}{Set of Quarternions}
        \coderow{\BB}{Set of Booleans}
        \midrule
        \coderow{\EE}{Expected value}
        \coderow{\Ave}{Average}
        \coderow{\Var}{Variance}
        \coderow{\Cov}{Covariance}
        \coderow{\Pr}{Probability}
        \midrule
        \coderow{\frac{x}{y}}{Fraction}
        \coderow{\sfrac{x}{y}}{Diagonal fraction}
        \coderow{\bsfrac{x}{y}}{Backwards diagonal fraction}
        \bottomrule
        \end{tabular}
    \end{table}

    \emph{Continued on the following page...}

    \begin{table}[H]
        \centering
        \begin{tabular}{@{}lcl@{}}
        \toprule
        Source Code & Result & Description \\ \midrule
        \coderow{\bigO}{Landau symbols (Big-O notation)}
        \coderow{\R}{Recursive languages}
        \coderow{\RE}{Recursively enumerable languages}
        \coderow{\coRE}{Complement-recursively enumerable languages}
        \coderow{\REC}{Recursively decidable languages}
        \coderow{\P}{Polynomial-time problems}
        \coderow{\NP}{Nondeterministic poly-time problems}
        \coderow{\coNP}{Complement-nondeterministic poly-time problems}
        \coderow{\DSpace}{Deterministic space class}
        \coderow{\NSpace}{Nondeterministic space class}
        \coderow{\DTime}{Deterministic time class}
        \coderow{\NTime}{Nondeterministic time class}
        \bottomrule
        \end{tabular}
    \end{table}

    Apart from these new macros, the template loads the \verb|amsmath| and \verb|amsthm| packages to provide the most relevant math macros and environments.

    \pagebreak
    \ex[Displaying Code]

    \subex[Real code]

    To render code into your submissions, you can use the features provided by \verb|minted|.
    Take the following python snippet.

    \begin{minted}{python}
        def fib(n):
            if n == 0 or n == 1:
                return 1
            return fib(n - 1) + fib(n - 2)
    \end{minted}

    You can render it using this code:
    \begin{minted}[escapeinside=||]{latex}
        \begin{minted}{python}
            def fib(n):
                if n == 0 or n == 1:
                    return 1
                return fib(n - 1) + fib(n - 2)
        \end||{minted}
    \end{minted}

    \subex[Pseudocode]
    To produce pseudocode, you can use the \texttt{pseudo} environment.

    \begin{multicols}{2}
        Take this sample program from the TCS lecture:
        \begin{pseudo}
            $P'$\;
            \eIf{$x_1 \neq 0$}{
                $x_0 \defeq x_0$
            }{
                \For{$x_0$}{
                    Do nothing...
                }
            }\;
            \If{$x_0 = 0$}{
                $x_0 \defeq 1$\;
                \While{$x_0 \neq 0$}{
                    $x_1 \defeq 1$
                }
            }
            \caption{Construction $P$}
        \end{pseudo}
    
        You can render it using this code:
        \begin{minted}[linenos=0]{latex}
            \begin{pseudo}
                $P'$\;
                \eIf{$x_1 \neq 0$}{
                    $x_0 \defeq x_0$
                }{
                    \For{$x_0$}{
                        Do nothing...
                    }
                }\;
                \If{$x_0 = 0$}{
                    $x_0 \defeq 1$\;
                    \While{$x_0 \neq 0$}{
                        $x_1 \defeq 1$
                    }
                }
                \caption{Construction $P$}
            \end{pseudo}
        \end{minted}
    \end{multicols}
    
    \pagebreak
    \ex[Drawing Automata]
    You can use TikZ to draw automata.
    See the following example:

    \begin{figure}[h]
        \centering

        \begin{tikzpicture}[node distance=3cm, auto]
            % Define states
            \node[state, initial] (s0)  {$S_0$};
            \node[state, below right = of s0] (s1)  {$S_1$};
            \node[state, above right = of s1] (s2)  {$S_2$};
            \node[state, accepting, above left = of s2] (s3)  {$S_3$};
            
            % Define transitions
            \path[->]
                (s0) edge [loop above] node {0} ()
                (s3) edge [loop left] node {0, 1} ()
                
                (s0) edge [bend right] node {1} (s1)
                (s1) edge [bend right] node {0} (s0)
                (s1) edge [swap] node {1} (s2)
                (s2) edge [swap] node {1} (s3)
                (s2) edge [swap] node {0} (s0)
            ;
        \end{tikzpicture}
    \end{figure}

    You can produce it using this code:
    \begin{minted}{latex}
        \begin{tikzpicture}[node distance=3cm, auto]
            % Define states
            \node[state, initial] (s0)  {$S_0$};
            \node[state, below right = of s0] (s1)  {$S_1$};
            \node[state, above right = of s1] (s2)  {$S_2$};
            \node[state, accepting, above left = of s2] (s3)  {$S_3$};
            
            % Define transitions
            \path[->]
                (s0) edge [loop above] node {0} ()
                (s3) edge [loop left] node {0, 1} ()
                
                (s0) edge [bend right] node {1} (s1)
                (s1) edge [bend right] node {0} (s0)
                (s1) edge [swap] node {1} (s2)
                (s2) edge [swap] node {1} (s3)
                (s2) edge [swap] node {0} (s0)
            ;
        \end{tikzpicture}
    \end{minted}

\end{document}