% !TeX root=Theoretical Computerscience Summary.tex
\documentclass[10pt,a4paper,english]{article}

\def\headersdir{../headers/}
\input{\headersdir udstasksheet.tex}

\lecture{\texttt{Theoretical Computerscience} -- Summary}
\author{Tim Schlachter (7039326)}
\term{WS 24/25}

\begin{document}
    \maketitle

    % Table of contents on separate page
    \newpage
    \tableofcontents
    \newpage

    % section about words
    \sectionwithtitle{Words}
A word \word (also called String) has length \length and consists of symbols \symbol \elem \sigmaS. \newline
The empty word \emptyword has length 0. \newline
    \newpage


    % section about regular languages
    \sectionwithtitle{Regular Languages}
    \newpage


    % section about regular expressions
    \sectionwithtitle{Regular Expressions}
    A regular expression always describes a regular language. If we can build a regular expression E, then L(E) \elem \REG. \newline

    \newpage


    % section about The Pumping Lemma
    \sectionwithtitle{The Pumping Lemma}
    \todo{Add chapter contents}
    \newpage


    % section about Equivalence Relations
    \sectionwithtitle{Equivalence Relations}
    \todo{Add chapter contents}
    \newpage


    % section about Myhill Nerode
    \sectionwithtitle{Myhill Nerode}
    \todo{Add chapter contents}
    \newpage


    % section about Limits of computability
    \sectionwithtitle{Limits of computability}
    \todo{Add chapter contents}
    \newpage


    % section about FOR and WHILE programs
    \sectionwithtitle{FOR and WHILE programs}

    \subsectionwithtitle{WHILE programming language}
        Variables: $x_0, x_1, \ldots, x_n$\newline
        Constants: $0, 1, 2, \ldots$\newline
        Keywords: while, do, od\newline
        Other symbols: $:=$,$\neq$, $;$, $+$, $-$, $[,]$\newline
        $W_0$: set of all simple statements\newline

    \subsectionwithtitle{FOR programming language}
        Variables: $x_0, x_1, \ldots, x_n$\newline
        Constants: $0, 1, 2, \ldots$\newline
        Keywords: for, do, od\newline
        Other symbols: $:=$,$\neq$, $;$, $+$, $-$, $[,]$\newline
    
    \subsectionwithtitle{Semantics}
        Input is stored in $x_0, \ldots, x_{s-1}$
        Output is the content of $x_0$ after execution of $P$.\newline
        The set $X = \{ x_0, x_1, x_2, \ldots \}$ if akk variables is finite.\newline

    \subsectionwithtitle{FOR/WHILE computable functions}
        \begin{enumerate}
            \item A partial function $f: \NN^s \rightarrow \NN$ is WHILE computable, if there is
            a WHILE program P such that $f = \phi_P$.
            \item f is FOR computable if $f = \phi_P$ for some FOR program P.
            \item The set of all WHILE computable functions is denoted by R.
            \item The set of all FOR computable functions is denoted by PR.
        \end{enumerate}

    \subsectionwithtitle{Decidable languages}
        The characteristic function of L is the function $\chi_L: \NN \rightarrow \NN$ defined by
        \begin{equation*}
            \chi_L(x) = \begin{cases}
                1 & \text{if } x \in L \\
                0 & \text{if } x \notin L
            \end{cases}
        \end{equation*}

        A language L is decidable if $\chi_L$ is computable ($\chi_L \elem R$).\newline
        The set of all decidable languages is denoted by \REC.\newline
    \newpage


    % section about Syntactic Sugar
    \sectionwithtitle{Syntactic Sugar}
    \todo{Add chapter contents}
    \newpage


    % section about Gödel numberings
    \sectionwithtitle{Gödel numberings}
    \todo{Add chapter contents}
    \newpage


    % section about Diagonalisation
    \sectionwithtitle{Diagonalisation}
    \todo{Add chapter contents}
    \newpage


    % section about Universal WHILE Program
    \sectionwithtitle{Universal WHILE Program}
    \todo{Add chapter contents}
    \newpage


    % section about Halting Problem
    \sectionwithtitle{Halting Problem}
    \todo{Add chapter contents}
    \newpage


    % section about Reductions
    \sectionwithtitle{Reductions}
    \subsectionwithtitle{Known problems}
    \subsubsectionwithtitle{Verification Problem $V$}
    The verification problem has two gödel number inputs namely $i, j$ and checks if the two programs corresponding to these gödel numbers output the same
    for every possible input.\newline

    \subsubsectionwithtitle{Special Verification Problem $V_0$}
    The special verification problem has a gödel number $i$ as input and checks if the program corresponding to that gödel number will output
    0 for every possible input.\newline

    \subsubsectionwithtitle{Program termination $T$}
    The program termination problem has a gödel number $i$ as input and checks if the program corresponding to that gödel number will terminate
    for every possible input.\newline
    \todo{Add chapter contents}

    \subsectionwithtitle{Many-one Reduction}
    A WHILE computable total function $f: \NN \rightarrow \NN$ is called many-one reduction from $L$ to $L'$ if
    \begin{itemize}
        \item $L, L' \subseteq \NN$
        \item $\forall x \elem \NN: x \elem L \iff f(x) \elem L'$
    \end{itemize}
    If such an $f$ exists, then $L$ is many-one reducible to $L'$, therefore we can write $L \leq L'$\newline

    \subsectionwithtitle{Known Reductions}
    $H_0 \leq V_0$\newline
    $V_0 \leq V$\newline
    $V_0 \leq T$\newline
    $\complement{H_0} \leq V_0$\newline
    \newpage

    % section about various proofs techniques
    \sectionwithtitle{Common Proof Techniques}
    \subsectionwithtitle{Pumping Lemma}
    The pumping lemma can only be used to show that a language is not \REG. We take a word \word \elem $L$ and split it into multiple sub words $u,v,w$
    where $\abs{v}\geq n$ with $n \ge 0$. Now there are words $x,y,z$ with $v=xyz$ and $\abs{y} \ge 0$ such that $uxy^izw$ \elem $L$ for all
    $i$ \elem \NN. Afterwards we try to find an $i$ such that the pumped word is no longer \elem $L$ and thus we proved that the language is not regular.

    \subsubsectionwithtitle{Example}
    \textbf{Exercise}:\newline
    Show that the following language is not regular.\newline
    Let $A=\{ 1^{(3n)} \mid n \elem \NN \}$\newline\newline
    \textbf{Solution}:\newline
    Let $n \geq 0$ be given.\newline
    We choose $u=1^n$, $v=1^n$, $w=1^n$ such that $uvw=1^{3n}$ and $\abs{v}=n$.\newline
    Let $x,y,z$ be given as $x=1^r$, $y=1^s$, $z=1^t$ with $xyz=v$ and $s \ge 0$ since $y \neq \emptyword$ and $r+s+t=n$.\newline
    $uxy^izw=1^n 1^r 1^{s \cdot i} 1^t 1^n=1^n 1^{r+s+t} 1^{s \cdot (i-1)} 1^n$\newline
    We choose $i=0$, therefore $1^n 1^{r+s+t} 1^{s \cdot (i-1)} 1^n = 1^n 1^{n-s} 1^n$ \notelem $A$ since it is not of the form $1^{3m}$ anymore for any $m$ \elem \NN.\newline
    Therefore we cannot pump language $A$ and thus it is not regular.\newline

    \subsectionwithtitle{Myhill Nerode}
    \subsubsectionwithtitle{Example}
    \todo{Add example for Myhill Nerode}

    \newpage

    \subsectionwithtitle{Reductions}
    \subsubsectionwithtitle{Example}
    \textbf{Exercise}:\newline
    Consider the following language $L = \{ i \elem \NN \mid \text{göd}^{-1}(i) \text{ outputs 42 on input 1337} \}$\newline
    Prove $L \notelem \coRE$\newline\newline
    \textbf{General Ideas}:\newline
    $\complement{L}$ is the set of all programs that either diverge or output something else than 42 on input 1337.\newline
    $L$ is the set of all programs that output 42 on input 1337 which is equal to running a sub program that halts on every input and outputting 42 afterwards.
    Thus we could try to reduce the special halting problem $H_0$ to $L$.\newline\newline
    \textbf{Solution}:\newline
    $\forall L: L \elem \REC \iff L \elem \RE \wedge \complement{L} \elem \RE$\newline
    $L \notelem \coRE \iff \complement{L} \elem \RE$\newline\newline
    $H_0 \notelem \REC$ but $H_0 \elem \RE$, therefore $\complement{H_0} \notelem \RE$ thus reducing $\complement{H_0} \leq \complement{L}$ suffices to show that
    $\complement{L} \notelem \RE$ but this is the same as $H_0 \leq L$ by equivalence.\newline\newline
    The reduction is given by $f(i) := \text{göd}(P)$\newline
    We create the WHILE program $P$, that has input $m$. However we ignore the input $m$ and simulate $i$ on the input $i$, then return 42.
    Thus $f$ is obviously WHILE computable.\newline\newline
    We consider the following two cases:
    \begin{itemize}
        \item Let $i \elem H_0$, then $i$ halts on input $i$ by definition. Program $P$ halts on arbitrary inputs m and outputs 42.
        Therefore $P$ also outputs 42 on our input 1337.\newline
        Thus $\text{göd}(P) \elem L$ holds.
        \item Let $i \notelem H_0$, then $i$ does not halt on input $i$ by definition. Therefore Program $P$ diverges on all inputs, which is also the case for input 1337.\newline
        Therefore $P$ diverges on our input 1337.\newline
        Thus $\text{göd}(P) \notelem L$ holds.
    \end{itemize}
    Therefore we have successfully shown that $H_0 \leq L$ is a valid reduction.
    \newpage

    % section about useful proofs
    \sectionwithtitle{Useful Proofs}

    \subsectionwithtitle{Regular Languages}

    \subsubsectionwithtitle{Finite Set}
    \textbf{Exercise}:\newline
    Show that the following language is regular over the alphabet $ \{0,1\} $.\newline
    $L=\{x \mid x \text{ is prime and } x < 1'000'000'000\}$\newline\newline
    \textbf{Solution}:\newline
    Since there are only finitely many prime numbers between $0$ and $1'000'000'000$, the set of the words that are accepted by $L$
    is finite and thus the language is regular.\newline


    \subsubsectionwithtitle{Finite Automaton}
    \textbf{Exercise}:\newline
    Show that the following language is regular over the alphabet $ \{0,1\} $.\newline
    $L=\{0^n 10^m \mid n,m \elem \NN \}$\newline\newline
    \textbf{Solution}:\newline
    \begin{figure}[h]
        \centering
        \begin{tikzpicture}[node distance=3cm, auto]
            % Define states
            \node[state, initial] (s0)  {$S_0$};
            \node[state, accepting, right=of s0] (s1)  {$S_1$};
            
            % Define transitions
            \path[->]
                (s0) edge [loop above] node {0} ()
                (s1) edge [loop above] node {0} ()
                
                (s0) edge [->] node {1} (s1)
            ;
        \end{tikzpicture}
    \end{figure}

    Since we can describe the language $L$ by the finite automaton given above, the language is regular.\newline


    \subsubsectionwithtitle{Regular Expression}
    \textbf{Exercise}:\newline
    Show that the following language is regular over the alphabet $ \{0,1\} $.\newline
    $L=\{0^n 1^m \mid n,m \elem \NN \}$\newline\newline
    \textbf{Solution}:\newline
    Let $E=0^* 1^*$ be the regular expression describing the language $L$.\newline
    Since we can describe the language $L$ by the regular expression given above, the language is regular.\newline


    \subsubsectionwithtitle{Closure Properties}


    \newpage
    \subsectionwithtitle{Non-Regular Languages}
    
    \subsubsection{Pumping Lemma}
    \textbf{Exercise}:\newline
    Show that the following language is not regular over the alphabet $ \{0,1\} $.\newline
    $L=\{0^n 1^n \mid n \elem \NN \}$\newline\newline
    \textbf{Solution}:\newline
    Let $n \geq 0$ be given.\newline
    We choose $u=1^n$, $v=1^n$, $w=1^n$ such that $uvw=1^{3n}$ and $\abs{v}=n$.\newline
    Let $x,y,z$ be given as $x=1^r$, $y=1^s$, $z=1^t$ with $xyz=v$ and $s \ge 0$ since $y \neq \emptyword$ and $r+s+t=n$.\newline
    $uxy^izw=1^n 1^r 1^{s \cdot i} 1^t 1^n=1^n 1^{r+s+t} 1^{s \cdot (i-1)} 1^n$\newline
    We choose $i=0$, therefore $1^n 1^{r+s+t} 1^{s \cdot (i-1)} 1^n = 1^n 1^{n-s} 1^n$ \notelem $A$ since it is not of the form $1^{3m}$ anymore for any $m$ \elem \NN.\newline
    Therefore we cannot pump language $A$ and thus it is not regular.\newline


    \subsubsection{Myhill Nerode}
    \textbf{Exercise}:\newline
    Show that the following language is not regular over the alphabet $ \{0,1\} $.\newline
    $L=\{0^n 1^n \mid n \elem \NN \}$\newline\newline
    \textbf{Solution}:\newline
    \todo{Add solution for Myhill Nerode}
    \newpage
    

    % Stichwortverzeichnis
    \newpage
    % Index soll Stichwortverzeichnis heissen
    % \renewcommand{\indexname}{Stichwortverzeichnis}
    % Stichwortverzeichnis soll im Inhaltsverzeichnis auftauchen
    \addcontentsline{toc}{section}{Index}
    % Stichwortverzeichnis endgueltig anzeigen
    \printindex


\end{document}