% !TeX root = Theoretical Computerscience Summary.tex
\documentclass[10pt,a4paper,english]{article}

\def\headersdir{../headers/}
\input{\headersdir udstasksheet.tex}

\lecture{\texttt{Theoretical Computerscience} -- Summary}
\author{Tim Schlachter (7039326)}

\begin{document}
    \maketitle

    % Table of contents on separate page
    \newpage
    \tableofcontents
    \newpage

    % section about words
    \sectionwithtitle{Words}
    A word \word (also called String) has length \length and consists of symbols \symbol \elem \sigmaS. \newline
    The empty word \emptyword has length 0. \newline


    % section about regular languages
    \newpage
    \sectionwithtitle{Regular Languages}


    % section about regular expressions
    \newpage
    \sectionwithtitle{Regular Expressions}
    A regular expression always describes a regular language. If we can build a regular expression E, then L(E) \elem \REG. \newline


    % section about various proofs techniques
    \newpage
    \sectionwithtitle{Common Proof Techniques}
    \subsectionwithtitle{Pumping Lemma}
    \subsubsectionwithtitle{Example}

    \subsectionwithtitle{Myhill Nerode}
    \subsubsectionwithtitle{Example}


    % section about useful proofs
    \newpage
    \sectionwithtitle{Useful Proofs}
    \subsectionwithtitle{Language is Regular}
    \subsubsectionwithtitle{Finite Set}
    \subsubsectionwithtitle{Finite Automaton}
    \subsubsectionwithtitle{Regular Expression}

    % Stichwortverzeichnis
    \newpage
    % Index soll Stichwortverzeichnis heissen
    % \renewcommand{\indexname}{Stichwortverzeichnis}
    % Stichwortverzeichnis soll im Inhaltsverzeichnis auftauchen
    \addcontentsline{toc}{section}{Index}
    % Stichwortverzeichnis endgueltig anzeigen
    \printindex


\end{document}