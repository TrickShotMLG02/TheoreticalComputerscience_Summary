% !TeX root=Theoretical Computerscience Summary.tex
\documentclass[10pt,a4paper,english]{article}

\def\headersdir{../headers/}
\input{\headersdir udstasksheet.tex}

\lecture{\texttt{Theoretical Computerscience} -- Summary}
\author{Tim Schlachter (7039326)}
\term{WS 24/25}

\begin{document}
    \maketitle

    % Table of contents on separate page
    \newpage
    \tableofcontents
    \newpage

    % section about words
    \sectionwithtitle{Words}
    A word \word (also called String) has length \length and consists of symbols \symbol \elem \sigmaS. \newline
    The empty word \emptyword has length 0. \newline


    % section about regular languages
    \newpage
    \sectionwithtitle{Regular Languages}


    % section about regular expressions
    \newpage
    \sectionwithtitle{Regular Expressions}
    A regular expression always describes a regular language. If we can build a regular expression E, then L(E) \elem \REG. \newline


    % section about various proofs techniques
    \newpage
    \sectionwithtitle{Common Proof Techniques}
    \subsectionwithtitle{Pumping Lemma}
    The pumping lemma can only be used to show that a language is not \REG. We take a word \word \elem $L$ and split it into multiple sub words $u,v,w$
    where $\abs{v}\geq n$ with $n \ge 0$. Now there are words $x,y,z$ with $v=xyz$ and $\abs{y} \ge 0$ such that $uxy^izw$ \elem $L$ for all
    $i$ \elem \NN. Afterwards we try to find an $i$ such that the pumped word is no longer \elem $L$ and thus we proved that the language is not regular.

    \subsubsectionwithtitle{Example}
    \textbf{Exercise}:\newline
    Show that the following language is not regular.\newline
    Let $A=\{ 1^{(3n)} \mid n \elem \NN \}$\newline\newline
    \textbf{Solution}:\newline
    Let $n \geq 0$ be given.\newline
    We choose $u=1^n$, $v=1^n$, $w=1^n$ such that $uvw=1^{3n}$ and $\abs{v}=n$.\newline
    Let $x,y,z$ be given as $x=1^r$, $y=1^s$, $z=1^t$ with $xyz=v$ and $s \ge 0$ since $y \neq \emptyword$ and $r+s+t=n$.\newline
    $uxy^izw=1^n 1^r 1^{s \cdot i} 1^t 1^n=1^n 1^{r+s+t} 1^{s \cdot (i-1)} 1^n$\newline
    We choose $i=0$, therefore $1^n 1^{r+s+t} 1^{s \cdot (i-1)} 1^n = 1^n 1^{n-s} 1^n$ \notelem $A$ since it is not of the form $1^{3m}$ anymore for any $m$ \elem \NN.\newline
    Therefore we cannot pump language $A$ and thus it is not regular.\newline

    \subsectionwithtitle{Myhill Nerode}
    \subsubsectionwithtitle{Example}


    % section about useful proofs
    \newpage
    \sectionwithtitle{Useful Proofs}

    \subsectionwithtitle{Regular Languages}

    \subsubsectionwithtitle{Finite Set}
    \textbf{Exercise}:\newline
    Show that the following language is regular over the alphabet $ \{0,1\} $.\newline
    $L=\{x \mid x \text{ is prime and } x < 1'000'000'000\}$\newline\newline
    \textbf{Solution}:\newline
    Since there are only finitely many prime numbers between $0$ and $1'000'000'000$, the set of the words that are accepted by $L$
    is finite and thus the language is regular.\newline


    \subsubsectionwithtitle{Finite Automaton}
    \textbf{Exercise}:\newline
    Show that the following language is regular over the alphabet $ \{0,1\} $.\newline
    $L=\{0^n 10^m \mid n,m \elem \NN \}$\newline\newline
    \textbf{Solution}:\newline
    \begin{figure}[h]
        \centering
        \begin{tikzpicture}[node distance=3cm, auto]
            % Define states
            \node[state, initial] (s0)  {$S_0$};
            \node[state, accepting, right=of s0] (s1)  {$S_1$};
            
            % Define transitions
            \path[->]
                (s0) edge [loop above] node {0} ()
                (s1) edge [loop above] node {0} ()
                
                (s0) edge [->] node {1} (s1)
            ;
        \end{tikzpicture}
    \end{figure}

    Since we can describe the language $L$ by the finite automaton given above, the language is regular.\newline


    \subsubsectionwithtitle{Regular Expression}


    \newpage
    \subsectionwithtitle{Non-Regular Languages}



    % Stichwortverzeichnis
    \newpage
    % Index soll Stichwortverzeichnis heissen
    % \renewcommand{\indexname}{Stichwortverzeichnis}
    % Stichwortverzeichnis soll im Inhaltsverzeichnis auftauchen
    \addcontentsline{toc}{section}{Index}
    % Stichwortverzeichnis endgueltig anzeigen
    \printindex


\end{document}