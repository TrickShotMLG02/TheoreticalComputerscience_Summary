% Mathe
\newcommand{\dif}{\mathop{}\!\mathrm{d}} % Differential - d
\newcommand{\im}{\mathop{}\!\mathrm{Im}\,} % Imaginäranteil
\newcommand{\re}{\mathop{}\!\mathrm{Re}\,} % Realanteil
\newcommand{\dgr}{^{\circ}} % Gradzeichen.
\newcommand{\degre}{\ensuremath{^\circ}} % Auch ein Gradzeichen, aber für GeoGebra-Exporte
\newcommand{\mvect}[1]{\left(\begin{matrix} #1 \end{matrix}\right)} % Vektor in Matrixschreibweise
\newcommand{\abs}[1]{\left| #1 \right|} % Setzt den Inhalt in Betragsstriche
\newcommand{\brk}[1]{\left( #1 \right)} % Setzt den Inhalt in große Klammern
\newcommand{\sbr}[1]{\left[ #1 \right]} % Setzt den Inhalt in eckige Klammern
\newcommand{\set}[1]{\left\lbrace #1 \right\rbrace} % Setzt den Inhalt in Mengenklammern (geschweifte Klammern)
\newcommand{\where}{\;\middle|\;} % Mengenbedingung { | }
\newcommand{\ceil}[1]{\left\lceil #1 \right\rceil} % Euklidklammern oben (aufrunden)
\newcommand{\floor}[1]{\left\lfloor #1 \right\rfloor} % Euklidklammern unten (abrunden)
\newcommand{\lequiv}{\Leftrightarrow} % Logisches äquivalent
\newcommand{\resetaligncount}{\setcounter{equation}{0}} % Setzt den Gleichungszähler zurück
\newcommand{\bdiv}{\mathop{\mathrm{div}}} % div operator for integer division

\newcommand{\NN}{\mathbb{N}} % Mengenzeichen: Natürliche Zahlen
\newcommand{\ZZ}{\mathbb{Z}} % Mengenzeichen: Ganze Zahlen
\newcommand{\QQ}{\mathbb{Q}} % Mengenzeichen: Rationale Zahlen
\newcommand{\RR}{\mathbb{R}} % Mengenzeichen: Reelle Zahlen
\newcommand{\CC}{\mathbb{C}} % Mengenzeichen: Komplexe Zahlen
\newcommand{\PP}{\mathbb{P}} % Mengenzeichen: Primzahlen
\newcommand{\HH}{\mathbb{H}} % Mengenzeichen: Quarternionen
\newcommand{\BB}{\mathbb{B}} % Mengenzeichen: Booleans

\newcommand{\tx}[1]{\cdot 10^{#1}} % Zehnerpotenzen

% Einheiten
\newcommand{\unit}[1]{\,\mathrm{#1}} % Einheit
\newcommand{\ufrac}[2]{\frac{\!\unit{#1}}{\!\unit{#2}}} % Einheitenbruch
\newcommand{\um}{\unit{m}} % Einheit: Meter
\newcommand{\us}{\unit{s}} % Einheit: Sekunde
\newcommand{\uh}{\unit{h}} % Einheit: Stunde
\newcommand{\uC}{\unit{C}} % Einheit: Coulomb
\newcommand{\uA}{\unit{A}} % Einheit: Ampère
\newcommand{\uN}{\unit{N}} % Einheit: Newton
\newcommand{\ukg}{\unit{kg}} % Einheit: Kilogramm
\newcommand{\uJ}{\unit{J}} % Einheit: Joule
\newcommand{\uW}{\unit{W}} % Einheit: Watt
\newcommand{\inv}{^{-1}} % Kehrbruch führ Einheiten (^-1)

% Theoretische Informatik
\newcommand{\bigO}{\mathcal{O}}
\newcommand{\R}{\mathsf{R}}
\newcommand{\RE}{\mathsf{RE}}
\newcommand{\REG}{\ensuremath{\mathsf{REG}}}
\newcommand{\coRE}{\mathsf{co\!-\!RE}}
\newcommand{\NRE}{\mathsf{NRE}}
\newcommand{\REC}{\mathsf{REC}}
\renewcommand{\P}{\mathsf{P}}
\newcommand{\NP}{\mathsf{NP}}
\newcommand{\coNP}{\mathsf{co\!-\!NP}}
\newcommand{\DSpace}{\mathsf{DSpace}}
\newcommand{\NSpace}{\mathsf{NSpace}}
\newcommand{\DTime}{\mathsf{DTime}}
\newcommand{\NTime}{\mathsf{NTime}}
\newcommand{\NL}{\mathsf{NL}}
\newcommand{\emptyword}{\ensuremath{\varepsilon}\xspace}    % For epsilon (you can also use \epsilon if you prefer)
\newcommand{\elem}{\ensuremath{\in}\xspace}                 % For element of symbol
\newcommand{\notelem}{\ensuremath{\notin}\xspace}           % For not element of symbol
% For sigma-related symbols
\newcommand{\sigmaS}{\ensuremath{\mathsf{\Sigma}}\xspace}   % For sigma in sans-serif font
\newcommand{\sumS}{\ensuremath{\sum}\xspace}                % For summation symbol
\newcommand{\sigmasub}{\ensuremath{\sigma}\xspace}          % For sigma (regular)
\newcommand{\notsigma}{\ensuremath{\not\in}\xspace}         % For not sigma (similar to not element)

\newcommand{\word}{\ensuremath{w}\xspace}                   % Representing a word w
\newcommand{\length}{\ensuremath{l}\xspace}                 % Representing a word length l
\renewcommand{\symbol}{\ensuremath{\sigma}\xspace}          % Representing a symbol sigma

% Statistik
\newcommand{\EE}{\mathbb{E}} % Erwartungswert
\newcommand{\Cov}{\mathrm{Cov}} % Covarianz
\newcommand{\Ave}{\mathrm{Ave}} % Mittelwert
\newcommand{\Var}{\mathrm{Var}} % Varianz
%\newcommand{\Pr}{\mathrm{Pr}} % Wahrscheinlichkeit (Probability)

% Zusammengesetzte Einheiten
\newcommand{\velo}{\ufrac{m}{s}} % Einheiten für Geschwindigkeiten
\newcommand{\accl}{\ufrac{m}{s^2}} % Einheiten für Beschleunigungen

% Rechenschritte in align
\newcommand{\ad}[1]{&& | \: #1}
\newcommand{\adt}[1]{&& | \: \text{#1}}

% Grabstein für Beweise
%\newcommand{\qed}{\hfill$\blacksquare$}

% Schönere Brüche
\newcommand{\sfrac}[2]{\nicefrac{#1}{#2}}
\newcommand{\bsfrac}[2]{\reflectbox{\nicefrac{\reflectbox{$#1$}}{\reflectbox{$#2$}}}}

% Vertically-centered colons
\newcommand{\vcol}{\vcentcolon}
\newcommand{\defeq}{\vcentcolon=}
\newcommand{\eqdef}{=\vcentcolon}
